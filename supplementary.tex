\section{Supplementary Material}

\subsection{Raw genomic input data sources  and genome assembly summaries}
\FloatBarrier

The yeast genomes and feature annotations were retrieved from the Saccharomyces Genome
Database \cite{cherry2011saccharomyces}. 

%need data also for Brassicaceae in these sections
The Brassicaceae genomes and feature annotations were retrieved from \textbf{XXXXX} (ref).

\begin{table}[hbpt]
  \centering
  \caption{Summary output for genomes assembly data. In this example table,
  \textbf{Scaffolds}  is a count of the number of scaffolds in each genome
  assembly; \textbf{Bases} is the total genome length; \textbf{Prots} is the
  total number of proteins annotated in the GFF; \textbf{GC} is the \%GC
  content; \textbf{Ns} is the total number of unknown bases (N) in the genome
  assembly. The species are listed in order of phylogenetic distance from
  \textit{S. cerevisiae}, the focal species.}
  \label{tab:genome-summaries}
  \begin{tabular}{lrrrrr}
    Species                  & Scafs &    Bases & Prots &    GC &     Ns \\
    \hline
    \textit{S. cerevisiae}   &    17 & 12157105 &  6008 & 0.381 &      0 \\
    \textit{S. paradoxus}    &   832 & 11872617 &  5933 & 0.387 &      0 \\
    \textit{S. mikatae}      &  1648 & 11470251 &  6086 & 0.380 &      0 \\
    \textit{S. kudriavzevii} &  2054 & 11189057 &  6529 & 0.398 &   2127 \\
    \textit{S. arboricola}   &    35 & 11619520 &  3659 & 0.387 & 224325 \\
    \textit{S. eubayanus}    &    24 & 11734173 &  5379 & 0.399 & 121986 \\
    \textit{S. uvarum}       &  1098 & 11477549 &  5721 & 0.402 &      0 \\
    % TODO: add nd50
  \end{tabular}
\end{table}

\FloatBarrier
\subsection{Synteny map construction and summaries}

The syntenic search intervals for yeast and Brassicaceae species against their respective focal species were inferred by \textbf{synder}, and calculated in [\textbf{SYNDER REF}] Summaries of this data are shown in \textbf{Table xx and xx}


\begin{table}[hbpt]
  \centering
  \caption{Numeric summary of the lengths of the syntenic blocks (in
  nucleotides) in the synteny maps between \textit{S. cerevisiae} and each of
  the listed target species. \textbf{N}, total number of blocks in the synteny map.}
  \label{tab:synmap-summary}
  \begin{tabular}{lrrrrrrrr}
    Species                  & min & q25 & median &    q75 &   max &     N \\
    \hline
    \textit{S. paradoxus}    &  63 & 177 &    389 & 852.00 & 13716 & 12232 \\
    \textit{S. mikatae}      &  63 &  80 &    275 & 552.25 &  6836 &  6688 \\
    \textit{S. kudriavzevii} &  65 &  75 &    257 & 489.00 &  4870 &  5189 \\
    \textit{S. arboricola}   &  65 &  71 &    159 & 417.00 &  4103 &  4534 \\
    \textit{S. eubayanus}    &  65 &  70 &    102 & 375.00 &  4751 &  3914 \\
    \textit{S. uvarum}       &  65 &  70 &    104 & 376.00 &  6836 &  3833 \\
  \end{tabular}
\end{table}

\FloatBarrier
\subsection{Synder results and summary}
\FloatBarrier

\begin{table}[!h]
  \centering
  \caption{Numeric summary of the lengths of the search intervals inferred by
  {\tt synder} between \textit{S. cerevisiae} and each of the listed target
  species.  The column 2-7 refer to the minimum, 25th quantile, median, 75th
  quantile, and maximum of the block lengths. The final column, \textbf{N}, is
  the total number of blocks in the synteny map.}
  \label{tab:synder-summary}
  \begin{tabular}{lrrrrrrrr}
    Species                   & min & q25    & median &    q75   &   max &     N \\
    \hline
    \textit{S. paradoxus}     &   1 &  670.0 &   1215 &  2211.00 & 26047 & 13219 \\
    \textit{S. mikatae}       &   1 &  556.5 &   1714 &  3858.00 & 31027 & 16039 \\
    \textit{S. kudriavzevii}  &   1 & 1006.0 &   2332 &  4919.75 & 26409 & 21460 \\
    \textit{S. arboricola}    &   1 & 1956.0 &   4643 &  9723.00 & 51035 & 19660 \\
    \textit{S. eubayanus}     &   3 & 2480.0 &   5967 & 12521.00 & 61711 & 23995 \\
    \textit{S. uvarum}        &   1 & 1327.5 &   3419 &  7746.00 & 45931 & 26095
  \end{tabular}
\end{table}

\begin{table}[!h]
  \centering
  \caption{Summary of {\tt synder} flags for Saccharomyces case study.
  \textbf{Inside} means that the query genes did not overlap any syntenic link
  in the gene (this does not mean the edges are ambiguous).  \textbf{Lo},
  \textbf{Hi}, \textbf{Both}, and \textbf{None} relate to whether the edges of
  the search interval are in a syntenically unambiguous region. A gene is
  counted as \textbf{Lo} if any search interval lower bound is unambiguous,
  \textbf{Hi} is any upper bound is unambiguous, \textbf{Both} if the lower and
  upper bound of any search interval is unambiguous, and \textbf{None} is no
  edge in any search interval is unambiguous.  \textbf{Scrambled} means all
  search intervals have ambiguous edges and are inbetween syntenic links.
  \textbf{Una} means the gene may be in an unassembled region of the genome
  (one edge of the search interval is flush against a terminus of a scaffold).}
  \label{tab:synder-flag-summary}
  \begin{tabular}{l | r | rrrr | r | r}
    Species                  & Inside & Lo   & Hi   & Both & None  & Scrambled & Una   \\
    \hline                                                         
    \textit{S. paradoxus}    & 699    & 5822 & 5814 & 5226 & 340   & 328        & 1038  \\
    \textit{S. mikatae}      & 3266   & 3980 & 4003 & 3370 & 2003  & 2027       & 3027  \\
    \textit{S. kudriavzevii} & 4135   & 2628 & 2655 & 2022 & 3312  & 3411       & 4568  \\
    \textit{S. arboricola}   & 4488   & 5137 & 5154 & 4856 & 1101  & 2149       & 513   \\
    \textit{S. eubayanus}    & 4972   & 4729 & 4724 & 4462 & 1541  & 2883       & 481   \\
    \textit{S. uvarum}       & 4987   & 2759 & 2754 & 2376 & 3430  & 3968       & 4256 
  \end{tabular}
\end{table}
% NOTE: this table is derived from the output of `synder::flag_summary`, but
% the columns have been renamed.
  % Inside    -- inbetween    = all(.data$inbetween),
  % Lo        -- lo_bound     = any(.data$l_flag < 2),
  % Hi        -- hi_bound     = any(.data$r_flag < 2),
  % Both      -- doubly_bound = any(.data$l_flag < 2 & .data$r_flag < 2),
  % None      -- unbound      = all(.data$l_flag > 1 & .data$r_flag > 1),
  % Scrambled -- incoherent   = any(.data$incoherent),
  % Una       -- unassembled  = any(.data$unassembled)

We infer the search intervals by calling the {\tt synder search} function. This
is a function of a synteny map and four parameters: 1) {\tt trans} specifies the
function needed to transform the score column in the synteny map to one that is
additive; 2) $k$ is the number of conflicting syntenic intervals allowed in a
block before it is broken (set to 0 by default); 3) $r$ is a score decay rate
that is used in calculating a score for each syntenic block created by {\tt
synder}; 4) {\tt offsets} which specify the input and output bases (0 or 1) of
the synteny map. Parameters 1 and 4 are specific to the tool that created the
synteny map. For the yeast study, where the synteny program {\tt MUMmer4} was used,
trans="p" (since the results are a percent identity) and offsets=11 since input
and output are 1-based.

The output of the {\tt synder search} function is 1) a set of one or more
search intervals for each focal gene (summarized in
\suptabref{tab:synder-summary}) and 2) a description of each seach interval
consisting of a flag describing each edge of the interval, whether the search
interval overlaps a syntenic region in the synteny map, and a relative score
for the search interval. {\tt synder} can then summarize all search intervals
for each focal gene into a single interpretation of the syntenic context of the
gene (\suptabref{tab:synder-flag-summary}).  

\subsection{Validation and parsing of GFF files}\label{subsec:cleaning-gffs}

The GFF format is simple but highly error prone and GFFs from difference
sources can follow very different conventions. This causes much difficulty in
practical analysis. While standardized formats exist for storing feature
information, GFF has persisted as the most commonly used. For this reason, we
choose to use GFF, but also carefully test the assumptions we make about the
content. The most problematic component of GFF is the 9th column that stores
tag-value data. It is from these tag-value pairs that we build the gene models
from which we extract the locations of features, the protein sequences for
genes, and the RNA sequences of transcripts.

\begin{description}
  \item[All columns have correct type] See \suptabref{tab:gff-types}
  \item[Unify type synonyms] According to the GFF3 specification
    (https://github.com/The-Sequence-Ontology/Specifications), names for the
    3rd column of a GFF3 file should contain the names or IDs of elements from
    the Sequence Ontology \cite{eilbeck2005sequence}. Based on this, we
    coalesce members from the following equivalence groups described in
    \suptabref{tab:sequence-ontology}.
  \item[Handle AUGUSTUS fields] The AUGUSTUS gene prediction program uses the
    tag 'Other' to represent the Parent relationship. If the {\tt source}
    column of the GFF3 file (2nd column) is "AUGUSTUS", then the tag 'Other'
    will be converted to 'Parent' (with a warning that will be passed through
    {\tt rmonad} to the user).
  \item[Assert that each Parent, ID, and Name tag contains a unique value] In
    the GFF3 spec, a tag can be associated with a comma-delimited list of
    values. {\tt fagin} currently does not support this and will raise an error
    if this case is found.
  \item[Treat Parent tags with a value of '-' as missing]
  \item[Handle unnamed fields] If no ID is given, but there is one untagged
    field, and if there are no other fields, then cast the untagged field as an
    ID. This is needed to accommodate the irregular output of AUGUSTUS.
  \item[Assert parent child relations are correct] \
  \begin{itemize}
    \item If a feature link to a Parent, then the Parent must exist in the GFF
    \item All Parent IDs must have either type 'gene' or 'mRNA'
    \item All CDS and all exon must have a Parent (gene or mRNA) 
    \item All mRNA and IDs must be unique
  \end{itemize}
\end{description}

\begin{table}[htpb]
  \centering
  \caption{A GFF file is a tab-delimited file with optional comments ('\#'
  initialized). The table must have nine columns, as listed under "Colum~name".
  The possible types of entries are listed under column "Base~type".  Entries
  in some columns are optional. The "phase" column is required for all GFF. The
  "attr" column contains a semicolon delimited list of 'tag=value' pairs.
  According to GFF3 specification, "attr" values may also be in comma delimited
  lists, but {\tt fagin} does not currently handle these lists and raises an
  error if a comma appears in one of the tags required by {\tt fagin} (Parent,
  ID, or Name).}
  \label{tab:gff-types}
  \begin{tabular}{llll}
    Column name & Base type & Optional & Notes and Restrictions \\
    \hline
    seqid  & string               & No  & all IDs present in genome \\
    source & string               & Yes & {}                        \\
    type   & string               & No  & {}                        \\
    start  & integer              & No  & {}                        \\
    end    & integer              & No  & $end \ge start$           \\
    score  & numeric              & Yes & {}                        \\
    strand & $+ \vert -$          & Yes & {}                        \\
    phase  & $0 \vert 1 \vert 2 $ & Yes & Required for CDS features \\
    attr   & tag-value list       & No  & {}                        \\
  \end{tabular}
\end{table}

\begin{table}[htpb]
  \centering
  \caption{Making the GFF files consistent: merging equivalent terms. The left
  column contains the names that all equivalent terms will be converted to. The
  $SO:XXXXXXX$ terms are Sequence Ontology IDs. {\tt fagin}  merges the $mRNA$
  and $transcript$ groups. These groups are technically different, but are
  often used interchangeably in practice. Also {\tt fagin} collapses the $exon$
  and the more specific $coding\_exon$ terms.}
  \label{tab:sequence-ontology}
  \begin{tabular}{ll}
    term & equivalent term \\
    \hline
    gene & SO:0000704 \\
    mRNA & messenger\_RNA , messenger RNA , SO:0000234, transcript, SO:0000673 \\
    CDS  & coding\_sequence , coding sequence, SO:0000316 \\
    exon & SO:0000147, coding\_exon , coding exon , SO:0000195
  \end{tabular}
\end{table}

\subsection{Extracting data from the
GFF files}\label{sec:derived-data}

{\bf Extracting mRNAs.} Given the GFF and genome, the extraction of the
transcripts by {\tt fagin} (mRNAs) is fairly straightforward.  The sequences of
all exons are extracted and concatenated. If the sense of the mRNA is negative,
the result is reverse transcribed.

\noindent
{\bf Extracting protein data.} Extracting the protein coding sequences from the
genome given the GFF is slightly more involved.  The GFF records all Coding
Sequences (CDS) and associates each with a parent (an mRNA or gene type
feature). The CDS may be spread across many exons, thus the CDS is a list of
intervals. These DNA intervals can be extracted from the genome and pasted
together to form the full CDS. However, there is some nuance to this step.
First, if the mRNA is negative sense, the CDS must be reverse transcribed. A
more difficult case arises when the initial interval of the CDS does not begin
in the correct reading frame.  This can happen, for example, when part of the
gene model is missing from the assembly. So the first interval in the CDS may
begin on the 2nd or 3rd position on the codon. The 'phase' column of the GFF
file stores the number of nucleotides that must be subtracted from the
beginning of a CDS interval to read the first complete codon. {\tt fagin}
stores the phase data and trims all models that start in a non-zero phase.
Thus, partial protein models are allowed.

Once the CDSs, or partial CDSs, have been extracted, they can be translated.
{\tt fagin} uses the {\tt translation} function from the {\tt Biostrings}
package of the Bioconductor project. By setting {\tt if.fuzzyi.codon="solve"},
it performs a "fuzzy" translation where codons with ambiguous nucleotides
(e.g., N for unknown base or Y for pyrimidine) will either be translated as X
(if more than one amino acid matches the pattern) or as a specific amino acid
(if only one amino acid matches). The resulting proteins are given the name of
their parent and stored for future use.

\noindent
{\bf Extracting ORFs in mRNAs and the genome.}

{\tt fagin} identifies ORFs in the mRNAs and across the entire genome. ORF
identification is limited to the mono-exonic case (i.e. splice sites are not
searched for). {\tt fagin}  uses the Bioconductor ORFik package \cite{orfik}.
This package is used to identify the longest, uninterrupted ORF for each stop
codon in the genome (or mRNA). For the genome (but not the mRNA) we search both
strands. The start and stop codons can be set by the user (the {\tt fagin}
default is {\tt START=ATG} and {\tt STOP={TAA,TGA,TAG}}). The minimum ORF
length can also be set by the user, with the default being 30 amino acids.

% TODO: I need to fix this in the code
{\tt fagin} currently uses the standard gene table for all genes. This would cause
problems in animal and fungi mitochondria and in several other cases (though
plant mitochondria and chloroplasts actually use the standard table).
